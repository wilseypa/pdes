
\documentstyle[11pt,alltt,url,fullpage,psfig]{article}

%\setlength{\unitlength}{1in}

\begin{document}

\title{\textbf{The \textsc{warped} Simulation Kernel's API Reference}} 

\author{ \textsf{Dhananjai Madhava Rao, Dale E. Martin,}\\
\textsf{Radharamanan Radhakrishnan, Malolan Chetlur,}\\ 
\textsf{and Philip A. Wilsey} \\
\textbf{Computer Architecture Design Laboratory} \\
\textbf{Dept. of ECECS, PO Box 210030} \\
\textbf{University of Cincinnati, Cincinnati, OH 45221--0030.}
}


\date{ }

\maketitle

\section{Introduction}

As far as the big picture goes, {\sc warped} is the specification of an
API of a discrete event simulator.  It defines interfaces required to
implement a discrete event simulation (both applications and new
simulation kernels).  In theory, the underlying implementation of the
discrete event simulation kernel should not be visible to the user.
Attempts have been made to ensure that this really is the case, and this
document will attempt to be thorough enough to allow the implantation of a
new ``plug and pray'' \textsc{warped} simulation kernel. To start things
off, Figure~\ref{fig:classHierarchy} illustrates the class hierarchy. The
kernel interface that the \textsc{warped} kernel builds off of is
described in the simulation developer's API reference
document~\cite{devel-api}.

\begin{figure}
\begin{center}
  \ \psfig{figure=figures/new-warped.eps,silent=}
\end{center}
\caption{Current Class Hierarchy}\label{fig:classHierarchy}
\end{figure}

\section{What a {\sc warped} kernel must supply}

\subsection{Kernel's view of LogicalProcesses}

This subsection defines how the logical process from the kernel's side is
organized.

\subsubsection{KernelLogicalProcessInterface}

The {\tt KernelLogicalProcessInterface} class defines what a kernel
process should provide to the user process. These methods are called from
the application. The {\tt LogicalProcess} class in turn routes the calls
via a kernel process pointer set by the configuration manager when the
object registers with it. A concise description of this class is shown in
Figure~\ref{fig:KernelLogicalProcessInterface} and a detailed description
of the interface is as follows.

\begin{figure*}
\begin{center}
\ \psfig{figure=figures/KernelLogicalProcessInterface.ps,silent=}
\end{center}
\caption{The interface kernel should provide to the user
  process.}\label{fig:KernelLogicalProcessInterface}
\end{figure*}

\begin{itemize}

\item{\tt void receiveUserEvent(UserEvent *)} - This method should receive
  the event passed to the kernel by the user and suitably process the
  event. Once the event has been handed over to the kernel, the user has
  no control over the data in the event and the user may not modify the
  contents of the event.

\item{\tt bool haveMoreEvents() const} - This method can be invoked to check
  if there are any events for this object waiting to be processed.  The
  values returned from haveMoreEvents() is valid only at that instant of
  time when the method is called. The value returned may not be valid
  across simulation cycles.
    
\item{\tt UserEvent *giveCurrentEvent() const} - This method returns the
  current pending event that is to be processed by the user process. The
  method return NULL if no pending events are present for this process.
  
\item{\tt const VTime *getSimulationTime() const} - This method should
  return the local virtual time of the process.
  
\item{\tt UserState* getState() const} - This method should return the
  current state of the process. The current state is valid once the
  simulation starts and before the process terminates. The function should
  return NULL when invoked during process creation and destruction. At all
  other times this function should return a non-null state pointer that
  points to a valid user state.

\item{\tt void processError(const char *message, const Severity} - This
  method provides access to exception handling in the system. The kernel
  process, depending on the severity level reported could abort, display a
  warning or log/ignore the message.

\end{itemize}

This subsection defines the interfaces that a simulation process presents
to the \emph{Cluster}.  A \emph{Cluster} is a collection of kernel logical
processes.  It provides a focal point and a place holder for all the
aggregate level sub-systems such as the scheduler, communication manager
and the GVT manager (what about the input, state, or the output
manager? They could be aggregate level sub-systems as well). 

\subsubsection{KernelLogicalProcess class}

The {\tt KernelLogicalProcess} class provides a handle to the user side
logical process. The {\tt Kernel\-LogicalProcess} class is inherited from
the {\tt KernelLogicalProcessInterface} class (see
Figure~\ref{fig:KernelLogicalProcessInterface}) and provides the similar
interface as the methods defined in the {\tt LogicalProcessInterface}
class. The cluster to which the kernel side process is attached calls
these methods and in turn the user defined method get called. The pointers
are initialized by the {\tt ConfigurationManager} class.

\subsubsection{StateManagerInterface class}

\begin{figure*}
\begin{center}
\ \psfig{figure=figures/StateManagerInterface.ps,silent=}
\end{center}
\caption{The interface every State Manager should provide to the kernel
  process.}\label{fig:StateManagerInterface}
\end{figure*}

The following subsection defines the interface for the state manager
associated with each kernel logical process. The state manager class
provides the necessary control needed to implement different state saving
strategies in the system.  Figure~\ref{fig:StateManagerInterface} provides
a concise description of the interface.

\begin{itemize}

\item {\tt KernelState *getCurrentState()} - This method is invoked by the
  kernel process to obtain its current state.

\item {\tt void saveState()} - This method is invoked by the kernel
  process at the end of each simulation cycle to indicate that a
  simulation cycle has been completed. The state manager may then decide
  whether to save state or not.  This provides the necessary information
  to implement different state saving mechanisms.

\item {\tt void rollback(const VTime *)} - This method is invoked by the
  kernel process to indicate that a rollback is occuring in the
  system. The state manager is expected to reset its internal data
  structures and restore the state to a simulation time that is passed
  in as a parameter.

\item {\tt void fossilCollect(const VTime *)} - This method is invoked by
  the kernel process to indicate that it is safe to fossil collect the
  state manager's internal data structures up to the simulation time that
  is passed in as a parameter.

\item {\tt void setLogicalProcess(KernelLogicalProcessInterface *)} - This
  method is invoked by the configuration manager to inform the state
  manager about the kernel process that is associated with it.

\item {\tt KernelLogicalProcessInterface *getLogicalProcess() const} -
  This method provides a convenient mechanism to access the process.
\end{itemize}

\subsubsection{OutputManagerInterface class}

\begin{figure*}
\begin{center}
\ \psfig{figure=figures/OutputManagerInterface.ps,silent=}
\end{center}
\caption{The interface an Output Manager should provide to the kernel
  process}\label{fig:OutputManagerInterface}
\end{figure*}

The following subsection defines the interface for the output manager
associated with each kernel logical process. The output manager class
provides the necessary control needed to implement different output
optimization strategies (such as lazy cancellation and dynamic
cancellation, one-anti message \emph{etc}.,).
Figure~\ref{fig:OutputManagerInterface} provides a consise description of
the interface. The following is a detailed description of the interface.

\begin{itemize}
\item {\tt void receiveUserEvent(UserEvent *)} - This method is invoked by 
  the kernel process to forward the events generated to the output
  manager. Inturn, the output manager forwards the events to the
  communication manager. 

\item {\tt void rollback(const VTime *)} - This method is invoked by the
  kernel process to indicate that a rollback is occuring in the
  system. The output manager is expected to reset its internal data
  structures and perform any necessary actions. 
  
\item {\tt void fossilCollect(const VTime *)} - This method is invoked by
  the kernel process to indicate that it is safe to fossil collect the
  output manager's internal data structures up to the simulation time,
  passed in as the parameter.

\item {\tt void setLogicalProcess(KernelLogicalProcessInterface *)} - This
  method is invoked by the configuration manager to inform the state
  manager about the kernel process that is associated with it. 

\item {\tt KernelLogicalProcessInterface *getLogicalProcess() const} -
  This method provides a convenient mechanism to access the kernel process
  associated with the output manager.

\end{itemize}

\subsubsection{Input Manager Interface class}

\begin{figure*}
\begin{center}
\ \psfig{figure=figures/InputManagerInterface.ps,silent=}
\end{center}
\caption{The interface every Input Manager should provide to the kernel
  process}\label{fig:InputManagerInterface}
\end{figure*}

The input manager class is a completely new facet of the redesign. The
input manager class may provide centralized input to the system. The input
manager may also handle the input messages that the scheduler passes down.
The input manager also provides mechanisms to implement some local
scheduling control and may also provide a source for applying more control
strategies.  Figure~\ref{fig:InputManagerInterface} provides a consise
description of the interface and the following is a detailed description
of the interface.

\begin{itemize}

\item {\tt void receiveUserEvent(UserEvent *)} - This method is invoked by
  the kernel process to schedule the events in the system.

\item {\tt void rollback(const VTime *)} - This method is invoked by the
  kernel process to indicate that a rollback is occuring in the
  system. The input manager is expected to reset its internal data
  structures and perform any necessary actions.
  
\item {\tt void fossilCollect(const VTime *)} - This method is invoked by
  the kernel process to indicate that it is safe to fossil collect the
  input manager's internal data structures up to the simulation time,
  passed in as the parameter.

\item {\tt void setLogicalProcess(KernelLogicalProcessInterface *)} - This
  method is invoked by the configuration manager to inform the state
  manager about the kernel process that is associated with it.

\item {\tt KernelLogicalProcessInterfcae *getLogicalProcess() const} -
  This method provides a convenient mechanism to access the process.
  
\item {\tt bool haveMoreEvents() const} - This method is invoked by the
  kernel process to ensure that the input manager has no pending events. 

\item {\tt UserEvent* giveCurrentEvent()} - This method is invoked by the
  kernel process to obtain any pending events from the local input
  manager.

\end{itemize}

These local manager classes provide the level of hierarchical control that
may be needed in the system.  These could be merely wrappers with inlined
functions (whose call that an optimizing compiler could optimize out) to
the corresponding queues in the system.  With this design, the symmetry of
the system is also maintained.  Now, the {\tt KernelLogicalProcess}
becomes a mere shell and a focal point for the different queues to attach
onto. This is analogous to the cluster that has a set of logical
processes; a logical process has a set of states and events.

\subsubsection{TimeWarpLogicalProcessInterface class}

\begin{figure*}
\begin{center}
\ \psfig{figure=figures/TimeWarpLogicalProcessInterface.ps,silent=}
\end{center}
\caption{The interface a TimeWarp logical process should provide to the
  cluster.}\label{fig:TimeWarpLogicalProcessInterface} 
\end{figure*}

The {\tt TimeWarpLogicalProcessInterface} class provides the interfaces
that should be provided by a TimeWarp-based logical process. The
Figure~\ref{fig:TimeWarpLogicalProcessInterface} provides a concise
description.  A more precise definition is given below.

\begin{itemize}
\item {\tt void rollback(const VTime *)} - This method is invoked by the
  cluster (or may be invoked from within the kernel side logical process
  itself) to indicate the occurrence of a rollback in the process. The
  time to which the process should rollback is specified by the input
  parameter. The process should roll back its local input queues to the
  specified time, send out anti-messages and restore its state to the
  appropriate state in the system.
  
\item {\tt fossilCollect(const VTime *)} - This method is invoked by the
  cluster (or may be invoked from within the kernel side logical process
  itself) to indicate that the process is free to fossil collect up to the
  time specified by the input parameter of the function. The process
  should fossil collect its input, state and output queues only at that
  point and fossil collection should be done only up to the time
  specified.

\item {\tt insertEvent(UserEvent *)} - This method is invoked by the
  cluster when an event has been received for that process. The kernel
  process may insert the event into its input queue or choose to ignore it.

\item {\tt setCluster(ClusterKernelLogicalProcessInterface *)} - This
  method will be called by the cluster when the {\tt ConfigurationManager}
  informs the cluster that a new simulation object is added to it. The
  cluster passes a pointer to itself as the parameter. The pointer should
  be saved for future references. 

\item {\tt ClusterKernelLogicalProcessInterface* getCluster() const} -
  This method provides access to the cluster to which this process 
  belongs. It should return the pointer provided using the {\tt
  setCluster} method.

\item {\tt void setStateManager(StateManagerInterface *)} - This method is
  invoked by the configuration manager to set the state manager associated
  with this process.

\item {\tt StateManagerInterface* getStateManager() const} - This method
  provides a convinient handle to the state manager. This method should
  return the pointer to the state manager set using the {\tt
  setStateManager} method.

\item {\tt void setInputManager(InputManagerInterface *)} - The input
  manager associated with the logical process is filled in by the
  configuration manager using this method.

\item {\tt InputManagerInteface* getInputManager() const} - This method
  should return the pointer set using the {\tt setInputManager()} method.
  
\item {\tt void setOutputManager(OutputManagerInterface *)} - This method
  is invoked by the configuration manager to inform the kernel logical
  process about the output manager associated with this process.

\item {\tt OutputManagerInterface* getOutputManager() const} - This
  method provides a convinient handle to the output manager.

\end{itemize}

\subsection{A TimeWarp cluster's interface}

This subsection defines the interface for a cluster used to maintain a set
of TimeWarp logical processes. The cluster provides a common point to tie
the different components such as the scheduler, GVT Manager, and the
Communication Manager together. Each of these sub components expects a
different set of information from the cluster. Hence a TimeWarp cluster's
interface is multiply inherited from the interfaces that each of the
sub-systems expected the cluster to provide. A simple sequential cluster
can be built by removing the communication manager and GVT Manager
interfaces. The following subsections define each of these sub-interfaces
in detail.

\subsubsection{A cluster's interface to a LogicalProcess}

\begin{figure*}
\begin{center}
\ \psfig{figure=figures/ClusterKernelLogicalProcessInterface.ps,silent=}
\end{center}
\caption{The interface a cluster should provide to a kernel logical
  process.}\label{fig:ClusterKernelLogicalProcessInterface}  
\end{figure*}

The class {\tt ClusterKernelLogicalProcessInterface} defines the methods
that a kernel should provide to the LogicalProcesses associated with
it. The cluster sees the methods defined in {\tt
KernelLogicalProcessIntercace} class (See
Figure~\ref{fig:KernelLogicalProcessInterface}) while the
KernelLogicalProcesses sees this interface to the
cluster. Figure~\ref{fig:ClusterKernelLogicalProcessInterface} provides a
concise description of the interface. Following is a detailed description
of the interface.

\begin{itemize}
\item {\tt sendUserEvent(UserEvent *)} This method is called by the
  KernelLogicalProcess to send events generated. This method is called
  by the kernel process when its event dispatch methods are invoked. This
  method should handle the event appropriately (send it across to the
  communication sub-system or schedule it if it is to a local guy).
\end{itemize}

\subsubsection{A cluster's interface to a GVT Manager}

\begin{figure*}
\begin{center}
\ \psfig{figure=figures/ClusterGVTManagerInterface.ps,silent=}
\end{center}
\caption{The interface a cluster should provide to a GVT
  manager.}\label{fig:ClusterGVTManagerInterface}   
\end{figure*}

The class {\tt ClusterGVTManagerInterface} defines the methods that any
GVT manager can invoke in the cluster.
Figure~\ref{fig:ClusterGVTManagerInterface} provides a concise description
of the interface.

\begin{itemize}
\item {\tt int getNumberOfLogicalProcesses() const} This method merely
  returns the number of logical processes that the cluster contains.

\item {\tt const VTime* getSimulationTime(int) const}: \\
  This method provides the GVT manager with a handle to the simulation
  time of a process. This method can be used by the GVT manager to merely
  iterate over the list of processes local to an cluster and estimate
  LGVT.

\item {\tt const VTime* getNthLogicalProcessSimulationTime(ObjectId)
const}:\\ 
  This method provides access to the simulation time of the object whose
  identification is passed as the parameter. 

\item {\tt void fossilCollect(const VTime *)} This method is called by
  the GVT manager to inform the cluster that it is safe to fossil collect
  up to the simulation time specified in the parameter. It is the
  responsibility of the cluster to further pass this information to the
  various logical processes and other sub-systems associated to it.
\end{itemize}

\subsubsection{A cluster's interface to the Configuration Manager}

\begin{figure*}
\begin{center}
\ \psfig{figure=figures/ClusterConfigurationManagerInterface.ps,silent=}
\end{center}
\caption{The interface a cluster should provide to a Configuration
  Manager.}\label{fig:ClusterConfigurationManagerInterface}    
\end{figure*}

The configuration manager is the final authority that decides the
configuration that the cluster is to operate under. The configuration
manager provides the cluster with all the sub-systems it is to use. The
class {\tt ClusterConfigurationManagerInterface} defines the methods that
the cluster must provide for the configuration manager.
Figure~\ref{fig:ClusterConfigurationManagerInterface} provides a concise
description.

\begin{itemize}
\item {\tt void setCommunicationManager(CommunicationManagerInterface *)}
  This method is invoked by the configuration manager to inform the
  cluster about the communication sub-system alloted to it.

\item {\tt CommunicationManagerInterface *getCommunicationManager() const}
  This method may be invoked to obtain the current communication system,
  both by the configuration manager (for inspection) and by the methods in
  the cluster for communication.

\item {\tt void setScheduler(SchedulerInterface *)} This method is used to
  set the scheduler associated with a cluster.

\item {\tt SchedulerInterface *getScheduler() const} This method is used
  to obtain a handle to the scheduler.

\item {\tt setGVTManager(GVTManagerInterface *)} This method is used to
  provide the cluster with an handle to the GVT manager.

\item {\tt GVTManagerInterface *getGVTManager() const} This method
  provides the handle to the GVT manager associated with the cluster.

\item {\tt void registerLogicalProcess(KernelLogicalProcessInterface *)}
  This method is one of the most important method that the cluster must
  provide to the configuration manager. This is through this method that
  logical processes are added to a cluster.
\end{itemize}

The methods to set and access the handles to the different sub-system is
provided here so that the configuration manager can obtain full control
over allocation and deallocation of the sub-systems. These methods can
also be used by the other methods used in the implementation of the
cluster.

\subsubsection{A cluster's interface to a scheduler}


\subsubsection{A cluster's interface to a CommunicationManager}

\begin{figure*}
\begin{center}
\ \psfig{figure=figures/ClusterCommunicationManagerInterface.ps,silent=}
\end{center}
\caption{The interface a cluster should provide to a communication 
  manager.}\label{fig:ClusterCommunicationManagerInterface}
\end{figure*}

Figure~\ref{fig:ClusterCommunicationManagerInterface} provides a brief
description of the interface a communication manager expects to see in a
cluster. A detailed description of the interface is as follows

\begin{itemize}
\item {\tt void receiveUserEvent(CommunicationEvent *)} This method is
  invoked by the CommunicationManager to inform the cluster that the
  communication sub-system received a user event and a pointer to the user
  event is provided via the communication event method.

\item {\tt void receiveKernelEvent(CommunicationEvent *)} This method is
  invoked to inform the cluster that a kernel message was received.
\end{itemize}

These methods have been provided to enable up calls in the system. The
communication sub-system could be a separate thread that invokes the
methods in the cluster.

\subsection{GVT Manager's interface}

The GVT Manager's interface consists of a set of interfaces that every GVT
manager should present to the different sub-systems in the simulator. The
different sub-systems being the cluster, communication manager and the
configuration manager. Figure~\ref{fig:GVTManagerInterface} presents a
concise description of the interface. Following is a detailed description
of the interface.

\begin{figure*}
\begin{center}
  \ \psfig{figure=figures/GVTManagerInterface.ps,silent=}
\end{center}
\caption{The interface a GVT Manager should
  provide.}\label{fig:GVTManagerInterface}
\end{figure*}

\begin{itemize}
\item {\tt void setCommunicationManager(CommunicationManagerInterface *)}
  This method is invoked by the configuration manager to inform the GVT
  manager about the communication sub-system associated with it. The GVT
  manager should save this pointer and use it for its communication.

\item {\tt CommunicationManagerInterface *getCommunicationManager() const}
  This method provides an convenient method to obtain an handle to the
  communication manager associated with it.

\item {\tt setCluster(ClusterGVTManagerInterface *)} This method is called
  by the configuration manager to inform the GVT manager about the cluster
  to which it is associated.

\item {\tt ClusterGVTManagerInterface *getCluster() const} This method
  provides an convenient mechanism to access the cluster associated with
  an GVT manager.

\item {\tt const VTime *getGVT() const} This method provides access to the
  most important information that the GVT manager provides to the system,
  the Global Virtual Time of the simulation.

\item {\tt void receiveGVTMessage(GVTMessage *)}:\\
  This method is called by the communication manager to inform the GVT
  manager about any GVT message tokens that it receives.

\item {\tt void inspectEvent(const UserEvent *)} This method is invoked
  from the communication manager to inform the GVT manager about the
  messages flowing in the system. Some of the GVT managers (Mattern for
  example) need to know about the events in the system for their working.
  This method may not modify the event passed to it.  It is the
  responsibility of the GVT manager to talk to the communication manager
  to inform the communication manager to forward all the messages to it
  (See CommunicationManagerInterface for details on this method). The GVT
  manager can inform the communication manager about its need to inspect
  events using the communication manager pointer passed to it by the
  configuration manager.
  
\end{itemize}
  
\subsection{Scheduler's interface}

A concise description of the interface a cluster expects its scheduler to
have is shown in Figure~\ref{fig:SchedulerInterface}. The following is a
detailed description of the interface.

\begin{figure*}
\begin{center}
  \ \psfig{figure=figures/SchedulerInterface.ps,silent=}
\end{center}
\caption{The interface a Scheduler should
  provide.}\label{fig:SchedulerInterface}
\end{figure*}

\begin{itemize}
\item {\tt void scheduleEvent(UserEvent *)} This method is called by the
  cluster that is employing this scheduler to schedule events on it. The
  scheduler is expected to schedule this event. The scheduler should
  handle all the associated overheads that may occur when a scheduler for
  TimeWarp is built. The scheduler should handle stragglers and
  anti-messages that could be passed down to it by the cluster. This
  method should be called only with non {\tt NULL} pointers.

\item {\tt UserEvent *giveEventToExecute()} This method is called by the
  cluster to determine which is the next event that has been scheduled for
  execution. The next event to be executed is completely determined by the
  scheduling policies adopted by the scheduler. If no events are
  available for scheduling this method return {\tt NULL}. But, a {\tt
    NULL} return value should not be interpreted as the queue being
  empty.

\item {\tt bool haveMoreEvents()} This method returns {\tt TRUE} if the
  scheduler has more events to process else it returns {\tt FALSE}. This
  method should be the one used by the cluster to decide if the scheduler
  queue is empty or not.

  The above two methods have been provided so that some blocking schemes
  can be implemented at the scheduler level.

\item {\tt void fossilCollect(const VTime *)} This method is called by
  the cluster to indicate to the scheduler that it is safe to fossil
  collect its internal data structures to the point in simulation time
  passed in as a parameter.
\end{itemize}

\subsection{Configuration Manager's interface}

\begin{figure*}
\begin{center}
  \ \psfig{figure=figures/ConfigurationManagerInterface.ps,silent=}
\end{center}
\caption{A Configuration Manager's
  interface.}\label{fig:ConfigurationManagerInterface}
\end{figure*}

The \emph{configuration manager} is the top level object that decides on
what the different sub-systems in the system and how they are
interconnected. Figure~\ref{fig:ConfigurationManagerInterface} defines the
interface every configuration manager should have. Following is a detailed
description of the interface.

The configuration manager is global to a given LP. A global pointer to the 
configuration manager is available to the different sub-systems that need
to interact with it.

\begin{itemize}
\item {\tt void initialize(int argc, char *argv[], char *env[], int\\
    maxClusterCount = 1)}: This method should be the first invoked method
  on the configuration manager. The configuration manager uses the
  parameters to determine the different configurations from the user
  sepcified parameters. It could use default values or report errors and
  abort the simulation. The {\tt maxClusterCount} indicates the maximum
  number of clusters that should be used. This number may be overriddedn
  by any of the other user defined parameters. The minimum value for this
  parameter is 1.

\item {\tt void registerLogialProcess(LogicalProcessInterface *, int
    clusterNumber = 0)}:\\ This method will be called by the user defined
  logical process to add a new process to a cluster. The cluster number is
  optional and defaults to cluster 0 (which is always present).

  It is at this point that configuration manager decides on the intial
  settings for the kernel side of the logical process. It instantiates the
  appropriate {\tt KernelLogicalProcess} (See
  Figure~\ref{fig:KernelLogicalProcessInterface} for more details), with
  the corresponding sub-systems and establishes the handles between them.
  It should also register the kernel side logical process with the
  corresponding cluster. The {\tt clusterNumber} specified by the user is
  optional and will be over-ridden by the configuration manager if it
  decides to.

\item {\tt void startSimulation()} This method should be called by the
  application process to start the simulation.
\end{itemize}

\subsection{Communication Manager's interface}

  SOON!!!

\bibliography{ref}
\bibliographystyle{acm}

\end{document}
